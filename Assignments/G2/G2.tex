\documentclass[11pt,twoside,a4paper]{article}

%========== PACKAGES ==========%

\usepackage{a4wide}				% for a wider layout
\usepackage{listings}			% for code examples
\usepackage{color}				% for color declarations
\usepackage{fancyhdr}
\usepackage{float}				% for the [H] position specifier
\usepackage[utf8]{inputenc}		% for UTF-8 encoding
\usepackage{lipsum}
\usepackage{mips}				% for MIPS syntax highlighting
\usepackage{multicol}			% for multicolumn layouts


%========== DEFINITIONS ==========%

\definecolor{c_comment}{rgb}	{0.38, 0.62, 0.38}
\definecolor{c_keyword}{rgb}	{0.10, 0.10, 0.81}
\definecolor{c_identifier}{rgb}	{0.00, 0.00, 0.00}
\definecolor{c_string}{rgb}		{0.50, 0.50, 0.50}


%========== MACROS ==========%

\newcommand{\figref[1]}{see Figure~\ref{#1}}	% use for referencing figures


%========== SETTINGS ==========%

\lstset
{
	numbers=left,
	frame=single,
	basicstyle=\footnotesize\ttfamily,
	tabsize=4,
	% colors
	commentstyle=\color{c_comment},
	keywordstyle=\color{c_keyword},
	identifierstyle=\color{c_identifier},
	stringstyle=\color{c_string},
}


%========== DECLARATIONS ==========%

\title
{
	\vspace{0.0in}	% if more space is needed for abstract/toc adjust this
	Machine Architecture\\
	\Large Assignment 2
}

\author
{
	Casper B. Hansen\\
	University of Copenhagen\\
	Department of Computer Science\\
	{\tt fvx507@alumni.ku.dk}
	\and
	Sine Vestergård Jensen\\
	University of Copenhagen\\
	Department of Computer Science\\
	{\tt kms698@alumni.ku.dk}
	\and
	Nikolaj Høyer\\
	University of Copenhagen\\
	Department of Computer Science\\
	{\tt ctl533@alumni.ku.dk}
}

\date{\today}

%========== DOCUMENT ==========%

\begin{document}

\clearpage
\maketitle
\thispagestyle{empty}
\hrule
\vspace{0.5in}	% spacing between title/authors and abstract/toc
\begin{multicols}{2}
\begin{abstract}
\lipsum[1]	% todo: write an abstract
\end{abstract}
\vfill
\columnbreak
\tableofcontents
\end{multicols}

\newpage
\pagestyle{fancy}
\section{Overview}
...

\subsection{Instruction set}
Our pipeline is quite simple, its instruction set
(\figref[fig:instruction-set]) consists of only 14 instructions.

\begin{figure}[H]
	\begin{multicols}{2}
	\center

	\textbf{Arithmetic-logical} \\
	\vspace{0.15in}
	\begin{tabular}{|l|l|l|l|}
		\hline
		\scriptsize {\bf Mnemonic} &
		\scriptsize {\bf Code} &
		\scriptsize {\bf Type} &
		\scriptsize {\bf Description} \\
		\hline {\tt addu} & {\tt 0x21} & {\tt R} & \scriptsize Add unsigned \\
		\hline {\tt addiu} & {\tt 0x09} & {\tt I} & \scriptsize Add imm. unsigned \\
		\hline {\tt slt} & {\tt 0x2A} & {\tt R} & \scriptsize Set less than \\
		\hline {\tt slti} & {\tt 0x0A} & {\tt I} & \scriptsize Set imm. less than \\
		\hline {\tt subu} & {\tt 0x23} & {\tt R} & \scriptsize Subtract unsigned \\
		\hline {\tt and} & {\tt 0x24} & {\tt R} & \scriptsize Logical AND \\
		\hline {\tt andi} & {\tt 0x0C} & {\tt I} & \scriptsize Logical imm. AND \\
		\hline {\tt or} & {\tt 0x25} & {\tt R} & \scriptsize Logical OR \\
		\hline {\tt ori} & {\tt 0x0D} & {\tt I} & \scriptsize Logical imm. OR \\
		\hline
	\end{tabular}
	
	\columnbreak
	\center

	\textbf{Memory-reference} \\
	\vspace{0.15in}
	\begin{tabular}{|l|l|l|l|}
		\hline
		\scriptsize {\bf Mnemonic} &
		\scriptsize {\bf Code} &
		\scriptsize {\bf Type} &
		\scriptsize {\bf Description} \\
		\hline {\tt lw} & {\tt 0x23} & {\tt I} & \scriptsize Load word \\
		\hline {\tt sw} & {\tt 0x2B} & {\tt I} & \scriptsize Store word \\
		\hline
	\end{tabular}

	\vspace{0.26in} % to align the bottoms of the tables
	\textbf{Branching} \\
	\vspace{0.15in}
	\begin{tabular}{|l|l|l|l|}
		\hline
		\scriptsize {\bf Mnemonic} &
		\scriptsize {\bf Code} &
		\scriptsize {\bf Type} &
		\scriptsize {\bf Description} \\
		\hline {\tt beq} & {\tt 0x04} & {\tt I} & \scriptsize Branch on equal \\
		\hline {\tt jal} & {\tt 0x03} & {\tt J} & \scriptsize Jump and link \\
		\hline {\tt jr} & {\tt 0x08} & {\tt R} & \scriptsize Jump to register \\
		\hline
	\end{tabular}

	\end{multicols}
	\caption{Instruction set}
	\label{fig:instruction-set}
\end{figure}

\subsection{Pipeline}
...

\newpage
\pagestyle{fancy}
\section{Preliminary design}
...

\newpage
\pagestyle{fancy}
\section{Tests}

\subsection{Arithmetic-logical operations}
...

\subsubsection{Forwarding}
\begin{multicols}{2}
\noindent ...
\vfill
\columnbreak
...
\lstset{language=[mips]Assembler}
\lstinputlisting{tests/arithmetic.asm}
...
\end{multicols}

\subsection{Memory-reference operations}
\begin{multicols}{2}
\noindent ...
\vfill
\columnbreak
...
\lstset{language=[mips]Assembler}
\lstinputlisting{tests/memref.asm}
...
\end{multicols}

\subsection{Branching operations}
...

\subsubsection{Hazards}
\begin{multicols}{2}
\noindent ...
\vfill
\columnbreak
...
\lstset{language=[mips]Assembler}
\lstinputlisting{tests/branching.asm}
...
\end{multicols}

\newpage
\pagestyle{fancy}
\begin{thebibliography}{9}

\bibitem{cod}
  David A. Patterson, John L. Hennessy,
  \emph{Computer Organization and Design}.
  Morgan Kaufmann,
  Revised 4th Edition,
  2009.

\end{thebibliography}

\end{document}
