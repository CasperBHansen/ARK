%
% logic.tex
%

\section{Digital Logic}
Here we list the different gates used to piece together circuits.

\usetikzlibrary{arrows,positioning}
\begin{multicols}{4}
	
	\begin{figure}[H]
		\center
%		\begin{tikzpicture}
%		[
%			align=center,
%			input/.style={rectangle,draw=black,fill=none},
%			output/.style={circle,draw=black,fill=none},
%			arrow/.style=
%			{
%				->,
%				thick,
%				shorten <=2pt,
%				shorten >=2pt,
%			},
%		]
%			\node[input] 	(A) at 	( 0.0, -0.5 ) {A};
%			\node[input] 	(B) at 	( 0.0,  0.5 ) {B};
%			\node[output] 	(C) at 	( 3.0,  0.0 ) {C};
%			
%			\draw ( 1.0, -0.5 ) to ( 2.0, -0.5 );
%			
%		\end{tikzpicture}
%		\\{\ }\\
		\begin{tabular}{c|cc}
			$\land$ & 0 & 1 \\ \hline
			0 & 0 & 0 \\
			1 & 0 & 1
		\end{tabular}
		\label{table:and-gate}
		\caption{Truth table of the AND-gate}
	\end{figure}
	
	\vfill
	\columnbreak
			
	\begin{figure}[H]
		\center
		\begin{tabular}{c|cc}
			$\lor$ & 0 & 1 \\ \hline
			0 & 0 & 1 \\
			1 & 1 & 1
		\end{tabular}
		\label{table:or-gate}
		\caption{Truth table of the OR-gate}
	\end{figure}
	
	\vfill
	\columnbreak

	\begin{figure}[H]
		\center
		\begin{tabular}{c|cc}
			$\neg\land$ & 0 & 1 \\ \hline
			0 & 1 & 1 \\
			1 & 1 & 0
		\end{tabular}
		\label{table:nand-gate}
		\caption{Truth table of the NAND-gate.}
	\end{figure}
	
	\vfill
	\columnbreak

	\begin{figure}[H]
		\center
		\begin{tabular}{c|c}
			$\neg$ & {\ } \\ \hline
			0 & 1 \\
			1 & 0
		\end{tabular}
		\label{table:not-gate}
		\caption{Truth table of the NOT-gate.}
	\end{figure}
	
\end{multicols}

