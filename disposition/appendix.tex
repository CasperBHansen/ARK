%
% appendix.tex
%

\begin{appendices}
	\section{Formulae}
	\subsection{Performance}
	\label{appendix:formulae|sub:performance}
	The relation between performance $P$ and execution time $E$ is defined as
	such
	\begin{align*}
		P_x = \frac{1}{E_x}
	\end{align*}
	Thus, performance is the reciprocal of the execution time. By this
	definition we can compare systems $x$ and $y$ by
	\begin{align*}
		P_x > P_y \quad\Rightarrow\quad \frac{1}{E_x} > \frac{1}{E_y}
	\end{align*}
	As is the case above, system $x$ has a higher performance than that of
	system $y$. We can also express this quantatively by
	\begin{align*}
		\frac{P_x}{P_y} = n
	\end{align*}
	which gives how much faster $x$ is compared to $y$. That is, the above
	equation states that systam $x$ is $n$ times faster than system $y$.
\end{appendices}

