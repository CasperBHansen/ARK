\documentclass[11pt,a4paper]{article}


%========== PACKAGES ==========%

\usepackage{a4wide} % save a few rainforests
\usepackage{amsmath,amssymb,amsthm} % for mathematics
\usepackage[toc,page]{appendix} % let's make it nice and neatly organized
\usepackage[utf8]{inputenc} % can i has utf-8, plx
\usepackage{multicol} % ...


%========== DEFINITIONS ==========%

\newcommand{\appendixref}[1]{(see appendix \ref{#1})}


%========== META ==========%

\title
{
	{\large Machine Architecture} \\
	Disposition	
}

\author
{
	Casper B. Hansen \\
	University of Copenhagen \\
	Department of Computer Science \\
	{\tt fvx507@alumni.ku.dk}
}


%========== DOCUMENT ==========%

\begin{document}

\clearpage\maketitle\thispagestyle{empty}
\begin{multicols}{2}
\abstract
{
	\noindent
	The purpose of this document is to summarize the core principles of the
	curriculum of the course {\it Machine Architecture}. It will contain brief
	answers to questions likely to be asked in the final exam set, as well as
	listings of formulae used to calculate different aspects of the machine.
}
\vfill
\columnbreak
\tableofcontents
\end{multicols}

\newpage
\section{Overview}
Here we will discuss the general principles. The topics covered here does
{\it not} go any further than the surface of the machine.

\subsection{Performance}
...

\begin{multicols}{2}
\subsubsection{Response time}
Also called {\it execution time}. The total time required for the computer to
complete a task, including disk accesses, memory accesses, I/O activities,
operating system overhead, CPU execution time, and so on.
\vfill
\columnbreak
\subsubsection{Throughput}
Also called {\it bandwidth}. Another measure of performance, it is the number
of tasks completed per unit of time.
\end{multicols}
For the equations of performance are listed under the appendix
\appendixref{appendix:performance}. 

\newpage
\begin{appendices}
	\section{Formulae}
	\subsection{Performance}
	\label{appendix:performance}
	The relation between performance $P$ and execution time $E$ is defined as
	such
	\begin{align}
		P(x) = \frac{1}{E(X)}
	\end{align}
	Thus, performance is the reciprocal of the execution time.
\end{appendices}

\end{document}

